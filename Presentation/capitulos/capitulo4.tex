\section{Experimentación}
\subsection{Modelo NR3DQA}
\begin{frame} 
  \frametitle{Introducción}
  \begin{enumerate}
    \item Para la validación hacemos uso de \textbf{validación cruzada ó \emph{K-fold}}.
      \begin{itemize}
        \item 9-fold en SJTU.
        \item 5-fold en WPC. 
        \item 11-fold en el biomédico.
      \end{itemize}
    \item Primeramente, \textbf{replicamos los resultados} de las publicaciones originales.
      \begin{itemize}
        \item Modelo NR3DQA.
        \item Modelo VQA-PC.
      \end{itemize}
    \item A continuación adaptamos los modelos. 
    \item Por último, \textbf{proponemos mejoras y analizamos los resultados}.
  \end{enumerate}
\end{frame}

\note{
Ya dentro de la experimentación, 
para la validación del modelo y la estimación de su rendimiento para las distintas 
mejoras propuestas hemos utilizado la técnica de  
validación cruzada, también conocida por K-fold. 
A modo resumen, a continuación replicamos los resultados obtenidos en 
las publicaciones originales de los métodos NR3DQA y VQA-PC. 
Para ello utilizamos los conjuntos de datos públicos SJTU y WPC. 
Por último, analizamos los resultados obtenidos sobre nuestro conjunto 
de imágenes médicas y, con ello, proponemos mejoras y adaptaciones.
}


\begin{frame}
  \frametitle{Modelo NR3DQA\footnotemark[9]}
\begin{table}[htp]
  \small
  \begin{center}
    \hspace{-.5cm}
    \begin{tabular}[c]{|c|c|c|c|c|}
      \hline
      \rowcolor[HTML]{FFC702}
      \multicolumn{1}{|c|}{\textbf{Dataset}} & 
      \multicolumn{1}{|c|}{\textbf{Modelo}} & 
      \multicolumn{1}{|c|}{\textbf{Escalado}} & 
      \multicolumn{1}{|c|}{\textbf{PLCC}} &
      \multicolumn{1}{|c|}{\textbf{SROCC}} \\
      \hline
      SJTU & SVM & MinMaxScaler & 0.810325 & 0.777403 \\ 
      \hline 
      WPC & SVM & MinMaxScaler & 0.637953 & 0.634853 \\
      \hline
      Biomédico & SVM & RobustScaler & 0.2017 & 0.1776 \\
      \hline
      Biomédico normalizado & KNNRegressor & RobustScaler & 0.2671 & 0.1882  \\
      \hline
      Biomédico en escala 0-5 & DecisionTree & StandardScaler & 0.309176 & 0.196713 \\
      \hline
    \end{tabular}
  \end{center}
  \caption[Resultados de prueba preliminar con NR3DQA.]{
    Resultados de prueba preliminar con NR3DQA\footnotemark[9]. 
  }
  \label{tab:MedicalNR3DQA}
\end{table}
\footnotetext[9]{\cite{NR3DQA}}
\end{frame}

\note{
Con el modelo NR3DQA seguimos el protocolo de experimentación especificado por sus 
autores. Logramos obtener resultados similares como se observan en las dos primeras filas.
No obstante, utilizando las mismas funciones para el caso de los imágenes médicas 
y experimentando con múltiples modelos de regresión y escalado de características, 
observamos, que el método no parece ser capaz
de determinar con precisión la calidad de dichas imágenes, 
ya que la correlación es cercana a 0.
En las últimas 3 filas podéis ver los mejores resultados, 
reafirmando así la dificultad de aplicar métodos IQA directamente a imágenes médicas.
}

\begin{frame}
  \frametitle{Modificaciones}
  \vspace{-.7cm}
  \begin{table}[htp]
    \small
    \begin{center}
    \caption[Resultado de mejoras sobre el método NR3DQA.]{Resultado de mejoras sobre el método NR3DQA.}
      \begin{tabular}[c]{|c|c|c|c|c|c|c|}
        \hline
        \rowcolor[HTML]{FFC702}
        \textbf{Dataset} & \textbf{Modelo} & \textbf{Escalado} & \textbf{PLCC} & \textbf{SROCC} \\ 
        \hline
        SJTU & SVM & MinMaxScaler & 0.853709 & 0.820057 \\ 
        \hline 
        WPC & SVM & MinMaxScaler & 0.642356 & 0.62917 \\
        \hline 
        Biomédico & SVM & StandardScaler & 0.344601 &  0.170793 \\
        \hline
        Biomédico en escala 0-5 & DecisionTree & StandardScaler & 0.30025  & 0.182296 \\
        \hline
      \end{tabular}
    \end{center}
    \label{tab:ImprovNR3DQA}
  \end{table}
  \begin{columns}
    \column{0.5\textwidth}
    \begin{enumerate}
      \item Weinmann et al\footnotemark ~estudiaron los procesos de: 
        \begin{itemize}
          \item Segmentación.
          \item Detección.
          \item Clasificación.
        \end{itemize}
    \end{enumerate}
    \column{0.5\textwidth}
    \begin{enumerate}
      \item Justifican la importancia de las características de:  
        \begin{itemize}
          \item Omnivarianza.
          \item Entropía de los valores singulares.
          \item Verticalidad del vecindario.
        \end{itemize}
    \end{enumerate}
\end{columns}
\footcitetext{3DNSSMetrics}
\end{frame}

\note{
Algo similar ocurre cuando intentamos utilizar más características geométricas
como se indica en el trabajo de Weinmann y sus colaboradores. Argumentan que 
en el proceso de segmentación, detección y clasificación de estructuras en nubes 
de puntos, las mejores características suelen ser: 
omnivarianza, entropía de valores singulares y la verticalidad del vecindario.
Esta adaptación logra un incremento del 5\% de correlación en SJTU llegando 
al 85\%, y de un 15\% en las imágenes médicas con la regresión por vectores soporte.
Sin embargo, sigue siendo apenas un 35\% de correlación.
El modelo adaptado de ML demuestra no ser capaz de determinar con calidad 
el nivel de distorsiones de nuestras imágenes médicas. 
}

\subsection{Modelo VQA-PC}

\begin{frame}
  \frametitle{Experimentos preliminares VQA-PC}
\begin{table}[htp]
  \small
  \begin{center}
    \begin{tabular}[c]{|c|c|c|}
      \hline
      \rowcolor[HTML]{FFC702}
      \textbf{Modelo} & \textbf{Dataset} & \textbf{SROCC} \\ 
      \hline
      VQA-PC & SJTU & \textbf{0.8813} \\ 
      \hline
      VQA-PC & WPC & \textbf{0.7181} \\
      \hline
    \end{tabular}
  \end{center}
  \caption[Resultados de experimento preliminar VQA-PC en SJTU.]{
    Resultados de experimento preliminar VQA-PC\footnotemark[10]. 
  }
  \label{tab:PreTestResults}
\end{table}
\begin{enumerate}
  \item Las predicciones del modelo no tienen por qué estar en la misma escala.
  \item Sería necesario utilizar regresión logística-5 para normalizar la escala. 
  \item Pero podemos centrarnos en el \textbf{SROCC} que es \textbf{invariante a la escala}.
\end{enumerate}
\footnotetext[10]{\cite{VQA-PC}}
\end{frame}

\note{
Para el modelo de DL hemos utilizado el mismo protocolo de validación. 
Cabe observar que los valores se acercan a los resultados obtenidos en la 
publicación original. 

Durante dicho procedimiento, hemos observado que el modelo no predice 
valores en la misma escala, sobre todo en el conjunto biomédico. 
Aun así, nuestro criterio de elección será la 
correlación entre las métricas. 
No necesitamos estimar los valores de distorsión 
en la misma escala que las etiquetas. Digamos que apenas necesitamos 
ser capaces de comparar de forma ordenada las imágenes de menor a mayor calidad. 
Con fin de ahorrar tener que normalizar dichas 
predicciones por medio de la regresión logística como propone Zhang y su colaboradores, 
nos centraremos en la métrica de correlación de SROCC por sus propiedades de invarianza 
a la escala. 
}

\begin{frame}
  \frametitle{Modificaciones VQA-PC}
\begin{table}[htp]
  \small
  \centering
\begin{tabular}{|c|cccc|}
\hline
\rowcolor[HTML]{FFC702}
                       & \multicolumn{4}{c|}{\textbf{Valor medio SROCC}}                                                                                                    \\ \hline
\rowcolor[HTML]{FFC702}
\textbf{Modelo}        & \multicolumn{1}{c|}{\textbf{Estándar}} & \multicolumn{1}{c|}{\textbf{Normalizado}} & \multicolumn{1}{c|}{\textbf{Reescalado}} & \textbf{Ambos}  \\ \hline
\textbf{VQA-PC (SJTU)} & \multicolumn{1}{c|}{0.7094}            & \multicolumn{1}{c|}{0.6235}      & \multicolumn{1}{c|}{\textbf{0.8425}}    & 0.7126          \\ \hline
\end{tabular}
\caption[Valor medio sobre imágenes médicas.]{Tabla de resultados iniciales sobre imágenes médicas.}
\label{tab:SroccMedRes}
\begin{enumerate}
    \item Experimentamos con \textbf{etiquetas normalizadas o no}.
    \item En vez de recortar una selección local, \textbf{reescalar la imagen entera}.
    \item Es evidente la \textbf{importancia del reescalado}.
  \end{enumerate}
\end{table}
\end{frame}

\note{
Dada la naturaleza de nuestras etiquetas, comparamos el modelo pre-entrenado 
sobre SJTU con las etiquetas estándares y sus versiones normalizadas por tipo 
de distorsión. Además, observamos que el modelo VQA-PC durante el entrenamiento 
recorta una zona aleatoria de la imagen a un tamaño 224x224. Dicho proceso, 
conocido como aumentación de datos incrementa el número de ejemplos y, habitualmente, 
previene el sobreajuste. Sin embargo, debido a que las nubes de puntos ocupan 
principalmente la zona central de la imagen y en poca proporción, dichos recortes 
pueden provocar ejemplos sin o con muy poca información. 
Esto puede ser perjudicial al modelo, es por ello que se propone un reescalado 
uniforme de la imagen entera.

Los resultados son prometedores. 
Con la implementación estándar, ya de por sí, logramos superar la línea del 70\% de correlación.
Vemos además, que normalizar las métricas es perjudicial. 
Y, sobre todo, destaca el reescalado de la imagen.
Para finalizar, nos centraremos en la característica principal de nuestro modelo. 
}

\begin{frame}
  \frametitle{Modificaciones VQA-PC}
  \begin{enumerate}
    \item Abouelaziz et al\footnotemark~ experimentaron \textbf{distintos métodos de fusión de características}.
      \begin{itemize}
        \item Fusión por \textbf{concatenación} (F0).
        \item Fusión por \textbf{multiplicación} (F1). 
        \item Fusión por \textbf{convolución 1x1} (F2).
        \item Fusión por \textbf{\emph{compact multi-linear pooling}} (F3).
      \end{itemize}
  \end{enumerate}
  \begin{columns}
    \column{0.5\textwidth}
\begin{table}[htp] 
  \scriptsize
  \centering
  \begin{tabular}{|c|c|c|c|}
\hline
\rowcolor[HTML]{FFC702}
                       & \multicolumn{3}{c|}{\textbf{SROCC}}                                                                                                          \\ \hline
\rowcolor[HTML]{FFC702}
\textbf{Modelo}        & \multicolumn{1}{c|}{\textbf{Media}} & \multicolumn{1}{c|}{\textbf{Desviación}} & \multicolumn{1}{c|}{\textbf{Mediana}} \\ \hline
\textbf{VQA-PC F0} & \multicolumn{1}{c|}{\textbf{0.8261}}   & \multicolumn{1}{c|}{0.1589}      & \multicolumn{1}{c|}{\textbf{0.8657}}      \\ \hline
\textbf{VQA-PC F1} & \multicolumn{1}{c|}{0.8164}   & \multicolumn{1}{c|}{0.1752}      & \multicolumn{1}{c|}{0.8637}      \\ \hline
\textbf{VQA-PC F2} & \multicolumn{1}{c|}{0.8057}   & \multicolumn{1}{c|}{0.1741}      & \multicolumn{1}{c|}{0.8538}      \\ \hline
\textbf{VQA-PC F3} & \multicolumn{1}{c|}{0.7482}   & \multicolumn{1}{c|}{\textbf{0.1326}}      & \multicolumn{1}{c|}{0.7518}      \\ \hline
  \end{tabular}
  \caption[Análisis de mejoras de fusión de características en VQA-PC sin pre-entrenar.]{
    Análisis de mejoras de fusión de características en VQA-PC sin pre-entrenar.
}
\label{tab:VQAFromScratch}
\end{table}
    \column{0.5\textwidth}
\begin{table}[htp]
  \scriptsize 
  \centering
\begin{tabular}{|c|c|c|c|}
\hline
\rowcolor[HTML]{FFC702}
                       & \multicolumn{3}{c|}{\textbf{SROCC}}                                                                                                          \\ \hline
\rowcolor[HTML]{FFC702}
\textbf{Modelo}    & \textbf{Media} & \textbf{Desviación} & \textbf{Mediana} \\ \hline
\textbf{VQA-PC F0} & 0.8325           & 0.2017              & 0.9140           \\ \hline
\textbf{VQA-PC F1} & 0.8242           & 0.2025              & 0.9095           \\ \hline
\textbf{VQA-PC F2} & \textbf{0.8757}  & \textbf{0.1468}     & \textbf{0.9347}  \\ \hline
\textbf{VQA-PC F3} & 0.8071           & 0.1811              & 0.8692           \\ \hline
\end{tabular}
\caption[Análisis de mejoras de fusión de características en VQA-PC pre-entrenado.]{
  Análisis de mejoras de fusión de características en VQA-PC pre-entrenado en LS-PCQA.
}\label{tab:VQAFromLSPCQA}
\end{table}

  \end{columns}
  \vspace{-.1cm}
  \footcitetext{EnsemblePCQA}
\end{frame}

\note{
Recordar que nuestro modelo intenta unificar las características estáticas con 
las dinámicas. Es por ello que tenemos una etapa esencial de fusión de características.
Experimentaremos modificar dicha etapa siguiendo el trabajo de Abouelazi y sus 
colaboradores. En la publicación hablan de los siguientes métodos: 
\begin{itemize}
\item El método habitual de fusión por concatenación F0
\item Métodos que permiten la interacción directa entre vectores características 
  como por ejemplo a través de multiplicación F1 o convolución F2.
\item Y, el más novedoso, la fusión, en el dominio de fourier, a través 
  compact multi-linear pooling F3
\end{itemize}
Comparamos los modelos sin pre-entrenar en contra de los 
modelos pre-entrenados sobre LS-PCQA, el conjunto de datos más grande. 
Ambos utilizando imágenes reescaladas. 
}

\note{
En la Tabla izquierda vemos que el método habitual F0 es el que obtiene  
mejor media y mediana de correlación. De esta forma, se determina que las mejoras 
F1-F3 no son significativas, al partir desde cero, para pequeños conjuntos de datos.
No obstante, en la tabla a la derecha con modelos pre-entrenados,
observamos que hemos logrado superar la línea del 80\% en todos.
Se observa una mejora significativa en los métodos F2 y F3, y se logra 
superar la barrera del 90\% en la mediana de los tres primeros métodos.
Gracias a la información adicional hemos logrado valores mucho mejores, llegando a 
obtener una correlación del 88\% de media utilizando el modelo F2 de fusión 
por convolución. 
}
