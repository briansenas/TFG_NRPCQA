\section[Introducción]{Introducción}
\subsection[Contexto]{Contexto}
\begin{frame}
    \frametitle{Contexto}
    \begin{columns}
      \column{0.5\textwidth}
    \begin{itemize}
      \item<1-> \textbf{La información visual} es cada vez más \textbf{importante}.
        \begin{itemize}
          \item<2-> Tanto para el entretenimiento como para el ámbito biomédico.
        \end{itemize}
      \item<3-> La estimación de calidad es la \alert<3->{tarea de medir y cuantificar} la calidad perceptual humana de una imagen (IQA). 
        \begin{itemize}
          \item<4-> Factores importantes: \textbf{contenido, contraste, distorsiones y la percepción humana}
        \end{itemize}
    \end{itemize}
    \column{0.5\textwidth}
    \begin{figure}
      \begin{center}
        \includegraphics[width=0.60\textwidth]{imagenes/chapter1/failure_minkowski_metric}
      \end{center}
      \caption{Imágenes distorsionadas equidistantes\footnotemark}
    \end{figure}
  \end{columns}
  \footcitetext{MinkowskiFailure}
\end{frame}

\note{
  \begin{enumerate}
    \item Con la demanda incremental de aplicaciones, tanto para el entretenimiento
como para el estudio biomédico, la información visual cada vez tiene un rol
más importante. Sin embargo, la calidad de dicha información puede verse
mermada con las etapas de adquisición, procesado, compresión, transmisión
y reproducción. Es por ello que poder evaluar dicha calidad se ha vuelto un
tema cada vez más importante.
\item 
En concreto, este Trabajo Fin de Grado (TFG) se centra en el estudio de la evaluación de la calidad de la imagen,
en inglés \emph{Image Quality assessment} (\emph{IQA}).
Es un problema fundamental en el procesamiento de imágenes y la visión 
por computador, que hace referencia a la tarea de medir y cuantificar 
la calidad perceptual de una imagen, 
teniendo en cuenta factores como el contenido, la resolución, 
el contraste, las distorsiones visuales y la percepción humana. 
La mejora de estas técnicas suele estar altamente conectada con el avance 
en los estudios del sistema de visión humano.
  \end{enumerate}
}

\subsection{Subproblemas}
\begin{frame}
  \frametitle{Estimación con referencia}
  \begin{columns}
  \column{0.5\textwidth}
    \begin{figure}
    \includegraphics[width=\textwidth]{imagenes/chapter1/FullReferenceInk.png}
    \caption{Problema con referencia (FR).}
    \end{figure}
  \column{0.5\textwidth}
    \begin{figure}
    \includegraphics[width=\textwidth]{imagenes/chapter1/ReducedReferenceInk.png}
    \caption{Problema con referencia reducida (RR).}
    \end{figure}
  \end{columns}
\end{frame}

\begin{frame}
  \frametitle{Estimación sin referencia}
  \begin{figure}
    \includegraphics[width=0.95\textwidth]{imagenes/chapter1/NoReferenceInk.png}
    \caption{Problema sin referencia (NR).}
  \end{figure}
  \begin{enumerate}[<+->]
    \item El subproblema más \textbf{difícil}.
    \item Debemos disponer de conocimientos sobre: 
      \begin{enumerate}
        \item Naturaleza de las imágenes.
        \item Efecto de las distorsiones.
      \end{enumerate}
    \item \textbf{Este TFG} aborda la estimación, \alert{sin referencia}, de calidad de imágenes médicas 3D.
  \end{enumerate}
\end{frame}


\subsection{Motivación}
\begin{frame}
  \frametitle{Aplicaciones}
  \begin{columns}
    \column{0.5\textwidth}
  \begin{enumerate}[<+->]
      \item\textbf{Comparativa} entre algoritmos de compresión.
      \item\textbf{Recuperación} de la información.
      \item\textbf{Evaluar} errores de transmisión.
  \end{enumerate}
  \column{0.5\textwidth}
  \begin{figure}
    \begin{center}
      \includegraphics[width=0.85\textwidth]{imagenes/chapter1/Brisque}
    \end{center}
    \caption{Eliminación de reflejos en imágenes\footnotemark~con medida de calidad BRISQUE\footnotemark}
  \end{figure}
  \end{columns}
  \footcitetext{BRISQUEExample}
  \footcitetext{BRISQUE}
\end{frame}


\begin{frame}
  \frametitle{Problemáticas}
  \begin{columns}
    \column{0.5\textwidth}
  \begin{itemize}
    \item El número de métodos propuestos para 3D \textbf{decrece sustancialmente}.
    \item<2-> La naturelaza de las imaǵenes médicas \textbf{reduce} la precisión de modelos IQA estándares.
    \item<3-> No hay \alert{ningún} método aplicado \alert{directamente} a imágenes médicas 3D.
  \end{itemize}
  \column{0.5\textwidth}
  \begin{figure}
    \begin{center}
      \includegraphics[width=0.80\textwidth]{imagenes/chapter1/MedicalDistortions}
    \end{center}
    \caption{Ejemplo de distorsiones médicas\footnotemark.}
  \end{figure}
  \end{columns}
  \footcitetext{MoreMedicalDistortion}
\end{frame}

\begin{frame}
  \frametitle{Motivación}
  \begin{columns}
    \column{0.5\textwidth}
    \begin{itemize}[<+->]
    \item Cada vez \alert{más frecuentemente} se emplean volúmenes tridimensionales.
    \item No obstante, las distorsiones \textbf{afectan al volumen 3D generado}. 
    \item Las contribuciones relativas al IQA en la medicina resulta en: 
      \begin{itemize}
        \item Reducción de costes. 
        \item Reducción de tiempo de consulta.
        \item Mejora de calidad del diagnóstico.
      \end{itemize}
    \end{itemize}
  \column{0.5\textwidth}
  \begin{figure}
    \begin{center}
      \includegraphics[width=0.95\textwidth]{imagenes/chapter1/SlicerVisualization}
    \end{center}
    \caption{Ejemplo de visualización 3D (Slicer\footnotemark).}
  \end{figure}
  \end{columns}
  \footcitetext{Slicer3D}
\end{frame}

\subsection{Objetivos}
\begin{frame}
  \frametitle{Objetivos}
  \begin{columns}

    \column{0.3\textwidth}
  \begin{enumerate}
    \item Estudió exhaustivo del estado del arte. 
    \item Generación de datos sintéticos.
    \item Validar métodos más prometedores.
  \end{enumerate}

  \column{0.7\textwidth}
  \vspace{-.5cm}
  \begin{figure}
      \includegraphics[width=1.0\textwidth, left]{imagenes/chapter1/Objetivos}
  \end{figure}

  \end{columns}
  
\end{frame}
