\chapter*{}
%\thispagestyle{empty}
%\cleardoublepage

%\thispagestyle{empty}

% \input{portada/portada_2}


%\cleardoublepage
\thispagestyle{empty}

\begin{center}
{\large\bfseries \myTitle}\\
\end{center}
\begin{center}
\myName\\
\end{center}

%\vspace{0.7cm}
\noindent{\textbf{Palabras clave}: 
modelos 3D, nubes de puntos, imágenes biomédicas, estimación de la calidad, aprendizaje profundo, visión por computador.
}\\

\noindent{\textbf{Resumen}}\\
En el campo biomédico, la visualización y análisis de estructuras 3D
desempeña un papel fundamental. 
Sin embargo, la calidad de estas representaciones puede variar debido a diversos 
factores, como la adquisición, procesamiento y reconstrucción de las estructuras anatómicas a analizar. 
Para mejorar cada uno de estos pasos se hace necesaria una manera de cuantificar las distorsiones que puedan surgir. 
Este campo de investigación todavía no ha sido suficientemente explorado en el ámbito biomédico.
\smallskip

Este TFG presenta un sistema capaz de estimar la calidad de 
una representación 3D biomédica sin referencia, es decir, sin emplear un objeto 
no distorsionado de referencia. La propuesta 
adapta modelos de estimación de calidad de nubes de puntos genéricas 
al ámbito médico. Dicha representación es la más común y versátil para 
estructuras complejas, reconstrucciones y segmentaciones en la medicina~\cite{WhyUsePointCloud}. 
\smallskip

De cara a la validación experimental, como no existen bases de datos públicas con imágenes 
biomédicas, se emplearon nubes de puntos generalistas, de personas, animales y objetos. 
De hecho, como parte de este TFG, se propone un conjunto de datos médicos sintéticos 
generados a partir de datos privados. 
Los datos consisten en un conjunto de tomografías computerizadas y modelos 
generados mediante el escaneo láser 3D de diferentes estructuras óseas. 
A este conjunto de datos se le aplican las distorsiones más comúnes 
en las representaciones médicas 3D (compresión, ruido y otras). 
El valor real de calidad, de cada ejemplo generado, es estimado por medio de 
los mejores métodos, para cada tipo de distorsión, según la literatura. 
\smallskip

A pesar de que los resultados obtenidos en este trabajo no dejan de ser preliminares, 
cabe mencionar que se alcanzó una correlación entre valores de calidad obtenidos y 
deseados del 88\%, sugiriendo que se trata de una línea de investigación tremendamente 
prometedora. 

\cleardoublepage


\thispagestyle{empty}


\begin{center}
{\large\bfseries \myTitleENG}\\
\end{center}
\begin{center}
\myName \\
\end{center}

%\vspace{0.7cm}
\noindent{\textbf{Keywords}: 
3D models, point clouds, medical images, quality assessment, deep learning, computer vision.
}\\

%\vspace{0.7cm}
\noindent{\textbf{Abstract}}\\

In the biomedical field, the visualization and analysis of 3D structures play a 
fundamental role. However, the quality of these representations can vary due to 
various factors, such as the acquisition, processing, and reconstruction of the 
anatomical structures to be analyzed. To improve each of these steps, a way to 
quantify the distortions that may arise is necessary. 
This research area has not yet been sufficiently explored in the biomedical domain.
\smallskip

This Bachelor's thesis presents a system capable of estimating the quality of a 
3D biomedical representation without reference, that is, without using an 
undistorted reference object. The proposal adapts quality estimation models from 
generic point clouds to the medical field. 
This representation is the most common and flexible for complex structures, 
reconstructions, and segmentations in medicine~\cite{WhyUsePointCloud}.
\smallskip

For experimental validation, as there are no public databases with biomedical images, 
generic point clouds of people, animals, and objects were used. 
In fact, as part of this Bachelor's thesis, a set of synthetic medical data generated 
from private data is proposed. The data consists of a collection 
of computerized tomographies and models generated through 3D laser scanning 
of different bone structures. This dataset is subjected to the most common 
distortions in 3D medical representations (compression, noise, and others). 
The real quality value of each generated example is estimated using the best methods for 
each type of distortion, according to the literature.
\smallskip

Although the results obtained in this work are still preliminary, 
it's worth mentioning that a correlation of 88\% was achieved between obtained 
quality values and desired values, suggesting that this is an extremely promising 
line of research.

\chapter*{}
\thispagestyle{empty}

\noindent\rule[-1ex]{\textwidth}{2pt}\\[4.5ex]

Yo, \textbf{\myName}, alumno de la titulación TITULACIÓN de la \textbf{\myFaculty}, 
con Pasaporte NX4L843F5, autorizo la ubicación de la siguiente copia de mi 
Trabajo Fin de Grado en la biblioteca del centro para que pueda ser
consultada por las personas que lo deseen.

\vspace{6cm}

\noindent Fdo: \myName

\vspace{2cm}

\begin{flushright}
Granada a \today.
\end{flushright}


\chapter*{}
\thispagestyle{empty}

\noindent\rule[-1ex]{\textwidth}{2pt}\\[4.5ex]

D. \textbf{\myProf}, Profesor del \myDepartment de la Universidad de Granada.

\vspace{0.5cm}

D. \textbf{\myOtherProf}, Profesor del \myDepartment de la Universidad de Granada.


\vspace{0.5cm}

\textbf{Informan:}

\vspace{0.5cm}

Que el presente trabajo, titulado \textit{\textbf{\myTitle}}
ha sido realizado bajo su supervisión por \textbf{\myName}, y autorizamos la defensa de dicho trabajo ante el tribunal
que corresponda.

\vspace{0.5cm}

Y para que conste, expiden y firman el presente informe en Granada a \today.

\vspace{1cm}

\textbf{Los directores:}

\vspace{5cm}

\noindent \textbf{\myProf \ \ \ \ \ \myOtherProf}

\chapter*{Agradecimientos}

\thispagestyle{empty}
       \vspace{1cm}


Primeramente, me gustaría agradecer a mis tutores, Enrique Bermejo y Pablo Mesejo,
por darme la oportunidad de desarrollar este proyecto con ellos. 
Agradezco la paciencia infinita y comprensión a la hora de resolver mis dudas. 
Segundo, agradezco a la propia Universidad de Granada por haberme dado la oportunidad 
de continuar mis estudios universitarios en tan distinguida casa de estudios. 
\\

Quiero agradecer también a mis padres, Joana Sena y Robert Netland, por la oportunidad 
de realizar mis estudios en España con todo sus apoyos. A mis compañeros de piso,
en especial al recién graduado en Física Yllari Kay, que han estado ahí desde 
primero de carrera y me han ayudado en cada paso. En general, a todos mis amigos, incluido los de Brasil,
por el cariño hacia mis obligadas ausencias. 
