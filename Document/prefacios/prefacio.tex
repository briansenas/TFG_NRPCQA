\chapter*{}
%\thispagestyle{empty}
%\cleardoublepage

%\thispagestyle{empty}

% \input{portada/portada_2}


%\cleardoublepage
\thispagestyle{empty}

\begin{center}
{\large\bfseries \myTitle}\\
\end{center}
\begin{center}
\myName\\
\end{center}

%\vspace{0.7cm}
\noindent{\textbf{Palabras clave}: 
aprendizaje automático, aprendizaje profundo, visión por computador, 
estimación de calidad, nubes de punto, imágenes médicas. }\\

\noindent{\textbf{Resumen}}\\
En el campo biomédico, la visualización y análisis de estructuras 3D, 
habitualmente nubes de puntos, desempeña un papel fundamental para el diagnóstico y investigación. 
Sin embargo, la calidad de estas representaciones pueden variar debido a diversos factores, como la adquisición, procesamiento y reconstrucción. 
Para mejorar cada uno de estos pasos es necesario una manera de cuantificar las distorsiones que puedan surgir. 
Ese problema se conoce como la estimación de calidad de nubes de puntos, 
un campo de especial y reciente importancia. No obstante, 
es un campo todavía sin explorar en el ámbito biomédico.
Para estimar la calidad de representaciones 2D o 3D, generalmente se extraen características de la imagen original y la distorsionada para su posterior 
comparación. Estas características provienen del análisis de estructuras, colores y 
del conocimiento del sistema visual humano. No obstante, 
muchas veces no disponemos de la información de referencia, como es el caso en 
la medicina, sino solo la versión distorsionada.
\\

Este TFG trata del desarrollo de un sistema capaz de estimar la calidad de 
una representación 3D biomédica sin referencia. La propuesta presenta la posibilidad de 
adaptación de modelos de estimación de calidad de nubes de puntos genéricas 
al ámbito médico haciendo uso de un meta-modelo de aprendizaje profundo para 
procesar proyecciones 2D multi-vista y un vídeo del objeto 3D rotando. 
De esta forma extraemos características tanto estáticas como dinámicas.
Para los experimentos se utilizaron diversos conjuntos de datos públicos 
de nubes de puntos genéricas, debido a que no existen en el caso biomédico, 
para la validación entre modelos y se propone un conjunto de datos médicos 
sintéticos generados a partir de datos privados. Aún siendo un conjunto de datos 
muy reducido, se logró una correlación entre los valores predicho y la 
estimación real de calidad del 86\% mostrando el potencial que posee esta línea de investigación.

\cleardoublepage


\thispagestyle{empty}


\begin{center}
{\large\bfseries \myTitleENG}\\
\end{center}
\begin{center}
\myName \\
\end{center}

%\vspace{0.7cm}
\noindent{\textbf{Keywords}: machine learning, deep learning, computer vision, quality assessment, point cloud, medical images.}\\

%\vspace{0.7cm}
\noindent{\textbf{Abstract}}\\

In the biomedical field, the visualization and analysis of 3D structures, 
typically point clouds, play a fundamental role in diagnosis and research. 
However, the quality of these representations can vary due to various factors, 
such as acquisition, processing, and reconstruction. 
To improve each of these steps, a way to quantify the distortions that may 
arise is necessary. This problem is known as point cloud quality estimation, 
a field of special and recent importance. However, it is still an unexplored 
field in the biomedical domain.
To estimate the quality of 2D or 3D representations, features are generally 
extracted from both the original and distorted images for subsequent comparison. 
These features come from the analysis of structures, colors, and knowledge of 
the human visual system. However, in many cases, we do not have the reference 
information, as is the case in medicine, but only the distorted version.
\\

This Bachelor's thesis deals with the development of a system capable of estimating 
the quality of a biomedical 3D representation without a reference. The proposal 
presents the possibility of adapting generic point cloud quality estimation models 
to the medical field by using a meta-model of deep learning to process multi-view 
2D projections and a video of the rotating 3D object. In this way, 
we extract both static and dynamic characteristics.
For the experiments, various public datasets of generic point clouds were used, 
as they do not exist in the biomedical case, for model validation. 
Additionally, a set of synthetic medical data generated from private data is proposed. 
Despite being a very limited dataset, a correlation of 86\% was achieved between 
the predicted values and the actual quality estimation, demonstrating the potential of this line of research.

\chapter*{}
\thispagestyle{empty}

\noindent\rule[-1ex]{\textwidth}{2pt}\\[4.5ex]

Yo, \textbf{\myName}, alumno de la titulación TITULACIÓN de la \textbf{\myFaculty}, 
con Pasaporte NX4L843F5, autorizo la ubicación de la siguiente copia de mi 
Trabajo Fin de Grado en la biblioteca del centro para que pueda ser
consultada por las personas que lo deseen.

\vspace{6cm}

\noindent Fdo: \myName

\vspace{2cm}

\begin{flushright}
Granada a 29 de julio de 2023.
\end{flushright}


\chapter*{}
\thispagestyle{empty}

\noindent\rule[-1ex]{\textwidth}{2pt}\\[4.5ex]

D. \textbf{\myProf}, Profesor del \myDepartment de la Universidad de Granada.

\vspace{0.5cm}

D. \textbf{\myOtherProf}, Profesor del \myDepartment de la Universidad de Granada.


\vspace{0.5cm}

\textbf{Informan:}

\vspace{0.5cm}

Que el presente trabajo, titulado \textit{\textbf{\myTitle, \mySubTitle}},
ha sido realizado bajo su supervisión por \textbf{\myName}, y autorizamos la defensa de dicho trabajo ante el tribunal
que corresponda.

\vspace{0.5cm}

Y para que conste, expiden y firman el presente informe en Granada a 29 de julio de 2023.

\vspace{1cm}

\textbf{Los directores:}

\vspace{5cm}

\noindent \textbf{\myProf \ \ \ \ \ \myOtherProf}

\chapter*{Agradecimientos}

\thispagestyle{empty}
       \vspace{1cm}


Primeramente, me gustaría agradecer a mis tutores, Enrique Bermejo y Pablo Mesejo,
por darme la oportunidad de desarrollar este proyecto con ellos. 
Agradezco la paciencia infinita y comprensión a la hora de resolver mis dudas. 
Segundo, agradezco a la propia Universidad de Granada por haberme dado la oportunidad 
de continuar mis estudios universitarios en tan distinguida casa de estudios. 
\\

Quiero agradecer también a mis padres, Joana Sena y Robert Netland, por la oportunidad 
de realizar mis estudios en España con todo sus apoyos. A mis compañeros de piso,
en especial al recién graduado en Física Yllari Kay, que han estado ahí desde 
primero de carrera y me han ayudado en cada paso. En general, a todos mis amigos, incluido los de Brasil,
por el cariño hacia mis obligadas ausencias. 
