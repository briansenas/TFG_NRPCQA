\chapter{Fundamentos Teóricos}
Esta sección tiene el objetivo de introducir y explicar brevemente los
fundamentos teóricos en los que se basan los métodos empleados en este
proyecto, así como justificar su relevancia en la resolución del problema que
se plantea.

\section{Aprendizaje Automático}
\towrite[caption={Fundamentos, Aprendizaje Automático}]{
  Cómo haremos uso de modelos de ML como SVR, KNNRegressor y entre otros, 
  así como el hecho de haber hecho ``feature selection'', análisis 
  PCA y de distribuciones creo que es oportuno empezar por este nivel 
  de introducción. 
  \\ 
  Sería: La definición, para qué casos sirve y terminar en porqué sería 
  un posible método para resolver este problema.  
  \\ 
  Creo que sería oportuno detenerme sobre todo con SVR y porqué es bueno 
  para alta dimensionalidad y como funciona a nivel del cálculo de margínes 
  y poner alguna imagen.
}

\section{Aprendizaje Profundo}
\towrite[caption={Fundamentos, Aprendizaje Profundo}]{
  Introducir las redes neuronales, hablar del aspecto de profundidad versus 
  anchura que presentan. Hablar de como se optimizan, backpropagation and feed-forward. 
  Hablar de las funciones de activación y los optimizadores.  
  Mencionar el proceso iterativo de entrenamiento, hablar del trade-off 
  bias-variance, mencioanr el problema de overfitting and underfitting 
}
\subsection{Redes Convolucionales} 
\towrite[caption={Fundamentos, Redes Convoluciones}]{
  Cómo el método ML principal actual es convoluciones, habrá que definir como 
  funciona las convoluciones. 
}
\todo[inline, shadow, caption={Importante}]{
  Quizás habría que unir subsecciones ó quizás hacer que fueran muy 
  breves ya que hay demasiadas. Pero comentar-las todas me parece 
  esencial para que el lector se ubique con los métodos empleados
}
\subsection{Convoluciones, \emph{Pooling} y \emph{BatchNorm}}
\subsection{Regularizaciones}
\subsection{Aplicadas a Videos} 
\section{Ensemble o Conjunto \emph{Deep Learning}}
\section{Representaciones 3D de las imágenes}
\section{Datos Sintéticos}
